% !TeX spellcheck = en_US
Gamification, especially in the context of education is nothing new and has been used by societies long before the modern era.
The concept of grading can be seen as a form of gamification, as it adds feedback and a competitive element to the learning process.
But in recent years especially with the advance of computers, gamification has become an increasingly popular topic in education science \parencite{swachaStateResearchGamification2021}.
Especially in computer science the use of gamified elements is well researched \parencite{dichevGamifyingEducationWhat2017}, which could be related to the already great use of computers in the field.
But as this field is still relatively new, many topics are still not well researched, especially how individual factors, such as gender, can influence the effects of gamification \parencite{dehghanzadehUsingGamificationSupport2024,oliveiraTailoredGamificationEducation2023}.
This thesis aims to explore the effects of different gamified elements and their combinations on performance and anxiety in a digital learning environment, especially focusing on the influence of gender on these effects.
Digital learning environments were hugely affected by the shift of teaching to remote classes during the COVID-19 pandemic from 