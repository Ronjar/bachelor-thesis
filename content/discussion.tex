\section{Discussion}
\subsection{Summary of Research focus}
\subsection{Implications}
\subsection{Limitations}
This study has several key limitations that must be acknowledged. Firstly, the reliance on self-reported data from students may lead to potential biases and inaccuracies in the findings. Previous research suggests that objective measures, such as sensor-based tracking, could provide more reliable data \parencite{woolfAffectiveTutorsAutomatic2010}.
Another significant limitation arises from the demographic and location constraints of our sample. The majority of participants were recruited from the computer science building on a technical campus, which may not provide a diverse representation of the general student population.
Additionally, the customization options for avatars were limited, lacking diverse representations such as Hijabs or beards, as noted by participants. This limitation may affect the engagement and identification of users with the avatars, potentially skewing the results regarding the impact of gamification on different genders.
The narrated content of the gamified elements could also be not engaging enough, being confined to a small portion of the screen without integrating characteristics, abilities, or dialogues that could enhance user interaction. This might have contributed to some participants opting to disable gamified elements, thus not experiencing the intended gamified environment fully.
Furthermore, the design of the study involved three iterations per participant, which might have led to habituation effects. These effects could influence the outcomes in terms of learning environment adaptability, anxiety, motivation, and self-efficacy, thus potentially diminishing the study's ability to measure these constructs accurately over time.
Finally, the lack of sufficient individualized feedback post-study was highlighted as a concern. Future iterations of this study should consider incorporating mechanisms for more personalized and actionable feedback to enhance the learning and adaptation process for participants.
Despite these limitations, the findings provide valuable insights into the intersection of gamification and gender, suggesting avenues for future research and practical applications in educational environments.

\subsection{Future Research}
\subsection{Conclusion}