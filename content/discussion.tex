\section{Discussion}
%\subsection{Summary of Research focus}
This thesis explored the effects of gamification elements on performance and anxiety in digital learning environments, with a focus on gender differences.
Using a randomized controlled design, the study assessed various gamification components like points, badges, leaderboards, and avatars, analysing responses from male and female participants.
The main goal was to identify how different genders react to these gamified elements regarding their cognitive and affective states.

Although the study did not find statistically significant effects, the data trends were in line with existing research, indicating consistent patterns in how participants especially of different gender reacted to the gamification elements.
This suggests that while the effects were not strong enough to be statistically significant, the observed behaviours reflected known theories and previous studies.
While the observed behaviours reflected known theories and previous studies, these effects did not reach significance.
\subsection{Interpretation}
These observed trends align with previous research that also noted similar patterns without definitive effects \parencite{dehghanzadehUsingGamificationSupport2024,hamariDoesGamificationWork2014}.
%This consistency suggests that while the direct impact of gamified elements may not be significant in our study, there is a consistent directional influence on learners' cognitive and affective states.
%This highlights the nuanced nature of gamification effects, suggesting that they may manifest under specific conditions or within certain contexts, as \textcite{dehghanzadehUsingGamificationSupport2024,koivistoRiseMotivationalInformation2019,oliveiraTailoredGamificationEducation2023}.
While the absence of the expected effect is noteworthy, it does not necessarily indicate that the effect does not exist.
For example maybe the estimated effect sizes are smaller than assumed and hence there were to less participants per condition to find these differences.
It's worth considering that the effect might be inherently more complex or context-dependent than originally hypothesized as also indicated by \textcite{dehghanzadehUsingGamificationSupport2024,koivistoRiseMotivationalInformation2019,oliveiraTailoredGamificationEducation2023}.
The failure to detect the effect might suggest that it only emerges under certain conditions or within particular subgroups, which our study design did not fully capture.
Additionally, the group of participants seems to be very homogenous, as the majority of students were bachelor students recruited from the computer science building.

\subsection{Implications}
Theoretically, the observed trends, hint at potential variations in how gamification impacts different learners, suggesting a need for broader conceptual models.
This could inspire further research into personalized learning environments.
Practically, these preliminary findings encourage a cautious approach to integrating gamification in educational settings.
Designers might consider more flexible gamification systems that can be adjusted based on gender, potentially enhancing user engagement without a one-size-fits-all strategy.
An example would be to implement leaderboards in male classes and use of avatars in female classes.


\subsection{Limitations}
This study has several key limitations that must be acknowledged.
Firstly, the reliance on self-reported data from students may lead to potential biases and inaccuracies in the findings. Previous research suggests that objective measures, such as sensor-based tracking, could provide more reliable data \parencite{woolfAffectiveTutorsAutomatic2010}.

Another significant limitation arises from the demographic and location constraints of our sample. The majority of participants were recruited from the computer science building on a technical campus, which may not provide a diverse representation of the general student population.

Additionally, the customization options for avatars were limited, lacking diverse representations such as Hijabs or beards, as noted by participants. This limitation may affect the engagement and identification of users with the avatars. It would have also been beneficial to allow users to personalize their avatars by entering their names, as this could further enhance user engagement and identification with the avatar.

The narrated content of the gamified elements could also be not engaging enough, being confined to a small portion of the screen without integrating characteristics, abilities, or additional dialogues that could enhance user interaction, while consuming additional time until the user could proceed. This might have contributed to some participants recalling to ignore the gamified elements altogether, thus not experiencing the intended gamified environment fully.
Furthermore, the design of the study involved three iterations per participant, which might have led to habituation effects. These effects could influence the outcomes in terms of learning environment adaptability, anxiety, motivation, and self-efficacy, thus potentially diminishing the study's ability to measure these constructs accurately over time.

Finally, the lack of sufficient individualized feedback by the participants post-study was highlighted as a concern. Future studies should consider incorporating mechanisms for feedback. This could provide valuable insights into the participants' experiences and perceptions of the gamified elements, potentially revealing more nuanced effects that were not captured in this study.
Despite these limitations, the findings provide valuable insights into the intersection of gamification and gender, suggesting avenues for future research and practical applications in educational environments.

\subsection{Future Research}
As highlighted in the introduction, the exploration of individual characteristics in the selection of gamified elements within learning environments requires further investigation.
This thesis has laid foundational insights into how gender influences engagement with gamified systems. 
However based on our results, the gamified elements showing descriptive trends, especially leaderboards and avatars in this study should be further explored.
For example different leaderboard styles with varying avatar integrations could be tested in order to mitigate gender differences.
As badges seem to have a negative outcome on anxiety, further research could explore the use of avatars in various implementations to counteract this effect for female students.
Other dimensions such as age, culture, and notably, the educational level (ranging from school-level to higher education and adult learning) merit in-depth exploration to comprehensively understand the dynamics at play.

Moreover, while the study's learning environment provided a basic platform, it does not directly represent the probable application of gamified elements. As ITS focus on supporting the learning process of the learner the effects of gamified elements on learning outcomes might be more pronounced using ITS.

Additionally, the temporal scope of this study was limited, addressing only short-term interactions. Prior research has established the significance of longitudinal studies in this domain \parencite{oliveiraTailoredGamificationEducation2023,dehghanzadehUsingGamificationSupport2024}. Extending research timelines to span multiple semesters would provide a more robust understanding of the long-term impacts of gamification on learning outcomes and student engagement. Such extended studies are quite challenging in every research area, but crucial for observing changes in learner stategies and behaviour and the sustainability of gamification benefits over time.

In conclusion, advancing research in these areas will not only enrich our understanding of how different groups respond to gamified learning but also enhance the design and implementation of educational technologies that are inclusive and effective across diverse learning environments.

\subsection*{Conclusion}
This study's exploration into the effects of gamification elements on performance and anxiety with a focus on gender differences has provided initial insights that, while not statistically significant, point to subtle yet potentially important trends.
The observed data suggest that certain gamified elements could have differential impacts on cognitive and affective outcomes across genders.
Although these effects were not strong enough to achieve statistical significance in this study, the descriptive findings hint at underlying patterns that merit further investigation.

The consistency of these patterns with prior research indicates that with improved research designs that address the limitations of this study, significant findings could be obtained.
These findings could be very important for advancing the understanding of how gamification can be optimized to enhance performance and anxiety effectively.
Future research should consider these aspects to uncover more robust evidence that can contribute to the ongoing discourse in educational technology.
