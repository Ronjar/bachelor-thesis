\section{Theoretical Background}

\subsection{Gamified digital learning environments}
Digital learning environments
Intelligent Tutoring Systems (ITS) can be referred to as "any computer program that can be used in learning and that contains intelligence" \>\parencite{freedmanLinksWhatIntelligent2000}.
Therefore, ITS accompany a student on his or her learning experience of a specific domain of knowledge by creating tasks that appeal to the students needs \parencite{gonzalezGamificationIntelligentTutoring2014}.
ITS can range from simple instructive texts to simulations and virtual realities, as a model imitating and abstracting certain aspects of the real world to reduce complexity for machine and user \parencite{psotkaIntelligentTutoringSystems1988}.
Modern ITS are split into three intertwined models: student, domain and tutor model. The student model holds the information about the user.
The user is characterized from different angles the information is updated as the user advances within the ITS.
The domain model contains the knowledge and structure of the learning material. It supplies the tutor model with tasks based on the input from the student model.
At last the tutor model controls the interaction with the student and contains the information which tasks are shown to the user in accord with the learning objectives from the domain model (\cite{gonzalezGamificationIntelligentTutoring2014}; \cite{freedmanLinksWhatIntelligent2000}).

With the rising interest in Gamification ITS also grow in importance as they can be used together. The use of gamified elements enhances the ITS and Gamification require some sort of progress tracking model for the gamified elements (e.g. Content unlocking to work).
\Textcite{gonzalezGamificationIntelligentTutoring2014} suggest adding more models as Gamification requires additional systems for visualization and feedback.
Additionally, new studies created a connection between emotions and learning, other studies connected relationships between teachers and students and an increase in student motivation \parencite{woolfAffectiveTutorsAutomatic2010}.
Both results were linked in a study by \Textcite{woolfAffectiveTutorsAutomatic2010} showing how digital learning companions improved the overall learning ability and self-concept of all students, especially low achieving students.
For the context of this thesis it is also important that the ITS version with a male learning companion was muted twice as much as the female version, resulting in a lower effect.
The study showed differences between gender, in discussion \Textcite{woolfAffectiveTutorsAutomatic2010} suggested considering gender within the student domain could further advance its predictive power.


\subsection{Gamification}
Gamification can be defined as "the idea of using game design elements in non-game contexts" \>\parencite{deterdingGameDesignElements2011} to further increase motivation and user activity within interaction design \parencite{deterdingGameDesignElements2011}.
These game-design elements "gamified elements" are elements often found in classical games. Often used elements are points, badges, leaderboards and avatars, other mechanisms include content unlocking, storytelling and memes \parencite{zainuddinImpactGamificationLearning2020}.
Often those elements are used specific constellations like the PBL triad described by \cite{werbachWinHowGame2012}, which contains points, badges and leaderboards, a system that is not only known from games, but also everyday enterprise features like loyalty programs and employee competitions \parencite{werbachWinHowGame2012}.
Points because they add an absolute scale, badges because they represent a status symbol and work like a temporary goal to strive toward and leaderboards to compare yourself to peers \parencite{werbachWinHowGame2012}.
One of the positive effects of gamification is brought by the feedback in different forms (task, process, self-regulation, self) either immediate or delayed. Feedback is one of the most important factors in the relation between education and learning \cite{sailerGamificationLearningMetaanalysis2020}.
The use of gamified elements showed positive outcomes in multiple studies, in general \parencite{hamariDoesGamificationWork2014} as well as in education specific contexts \parencite{sailerGamificationLearningMetaanalysis2020}.
But gamification, especially some elements like leaderboards, can also lead to negative outcomes. Leaderboards, while motivating through comparison, have been reported to demotivate participants \parencite{almeidaSystematicMappingNegative2021}.
"Pavlovication" as \cite{klabbersArchitectureGameScience2018} calls it, Gamification, as it is often a short question-answer-reward-cycle, conditions the user to learn conditional and narrows the possible ways to solve a problem down ~\parencite{klabbersArchitectureGameScience2018}.
Some studies also suggested that gamified learning platforms also lack individualism regarding choice and display of gamification elements, resulting in discomfort and negative emotions \parencite{santosDoesGenderStereotype2023}.

\subsection{Gender and Stereotype threat}
Gender, as a concept within social sciences, refers to more than the binary categorization of male and female.
It encompasses a range of identities and experiences that are shaped by a complex interplay of biological, psychological, and social factors.
Gender is not solely determined by biological characteristics; instead, it is increasingly recognized as a spectrum, acknowledging the presence of diverse gender identities beyond the traditional binary understanding \parencite{lindqvistWhatGenderAnyway2021}.
Socialization plays a critical role in shaping gender identity. It influences how individuals perceive themselves and interact with their surroundings based on the gender norms prevalent within their society.
These norms dictate behaviors, roles, and expectations, which are often internalized from an early age through various socialization agents like family, media, educational institutions, and peer groups \parencite{kampshoffHandbuchGeschlechterforschungUnd2012}.
While acknowledging the spectrum of gender identities, this thesis will focus primarily on the binary categorization of gender—male and female.
This approach does not negate the validity of non-binary or genderqueer identities but rather limits the scope of investigation to traditional gender roles within the binary framework.

Stereotype threat occurs when "one can be judged by, treated in terms of, or self-fulfill negative stereotypes about one's group".
Although this study does not aim to eliminate stereotype threat it is an important factor as it can explain at least some of the differences different genders experience while studying computer science \parencite{cheryanClassroomsMatterDesign2011}, especially regarding math \parencite{spencerStereotypeThreatWomen1999}.
Stereotype threat even leads to lower identification with academics and specific subjects \parencite{christyLeaderboardsVirtualClassroom2014}.