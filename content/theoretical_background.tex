\section{Theoretical Background}

\subsection{Tutorial systems}
Tutorial systems have become increasingly popular in educational settings, offering a wide range of features and flexibility to support students learning processes.
These systems can range from simple instructive texts to simulations and virtual realities, serving as models that simplify aspects of the real world to reduce complexity mostly from interconnection and context of knowledge for both the machine and the user \parencite{psotkaIntelligentTutoringSystems1988}.
Those Environments are in use at higher education institutions and since the COVID-19 pandemic, they have become more important \parencite{elhadbiReviewStudyAdaptive2024} and widely researched, especially in the field of computer science \parencite{zawacki-richterSystematicReviewResearch2019}.

Tutorial systems are often enhanced with some sort of intelligence (ITS). A dynamic adaptation to learner's needs, incorporating usage factors such as performance, but also external factors such as age, culture or gender \parencite{nkambouAdvancesIntelligentTutoring2010, gonzalezGamificationIntelligentTutoring2014}.
The role of Artificial Intelligence in this process could also become more important in the future, as it could have a significant, yet unknown impact on ITS \parencite{zawacki-richterSystematicReviewResearch2019}.

Tutoring Systems are often used in combination with gamified elements, significantly improving the learning experience \parencite{dermevalGaTOOntologicalModel2019}.
\textcite{jacksonMotivationPerformanceGamebased2013} found that the use of gamified elements in tutoring systems significantly improved the motivation and performance of students compared to traditional tutoring systems which lead to boredom and disengagement after long periods of use.

Incorporating gamified elements not only enhances the engagement and motivation within the ITS but also necessitates mechanisms for tracking progress, such as content unlocking \parencite{gonzalezGamificationIntelligentTutoring2014}.
The evolving landscape of ITS research also includes emotional and relational dynamics, linking student emotions and teacher-student relationships to learning efficacy and motivation \parencite{woolfAffectiveTutorsAutomatic2010}.
These insights have led to the development of digital companions, often named pedagogical agents, within ITS that significantly boost the learning potential and self-concept of students, particularly those who are low-achieving.
Intriguingly, a study noted that ITS programs with a male companion were muted twice as often as those with a female companion, highlighting potential gender differences that could be explored to enhance the predictive capabilities of the student model \parencite{woolfAffectiveTutorsAutomatic2010}.

\subsection{Gender and Stereotype threat}
Gender, as a concept within social sciences, refers to more than the binary categorization of male and female.
It encompasses a range of identities and experiences that are shaped by a complex interplay of biological, psychological, and social factors.
Gender is not solely determined by biological characteristics; instead, it is increasingly recognized as a spectrum, acknowledging the presence of diverse gender identities beyond the traditional binary understanding \parencite{lindqvistWhatGenderAnyway2021}.
Socialization plays a critical role in shaping gender identity. It influences how individuals perceive themselves and interact with their surroundings based on the gender norms prevalent within their society.
These norms dictate behaviours, roles, and expectations, which are often internalized from an early age through various socialization agents like family, media, educational institutions, and peer groups \parencite{kampshoffHandbuchGeschlechterforschungUnd2012}.
While acknowledging the spectrum of gender identities, this thesis will focus primarily on the binary categorization of gender—male and female.
This approach does not negate the validity of non-binary or genderqueer identities but rather limits the scope of investigation to traditional gender roles within the binary framework.

Stereotype threat occurs when "one can be judged by, treated in terms of, or self-fulfill negative stereotypes about one's group".
Although this study does not aim to eliminate stereotype threat it is an important factor as it can explain at least some differences different genders experience while studying computer science \parencite{cheryanClassroomsMatterDesign2011}, especially regarding math \parencite{spencerStereotypeThreatWomen1999}.
Stereotype threat even leads to lower identification with academics and specific subjects \parencite{christyLeaderboardsVirtualClassroom2014}.

\subsection{Gamification}
Gamification can be defined as "the idea of using game design elements in non-game contexts" \>\parencite{deterdingGameDesignElements2011} to further increase motivation and user activity within interaction design \parencite{deterdingGameDesignElements2011}.
These game-design elements, subsequently called gamified elements, are elements often found in classical video games. However, the concept of gamification is different from designing a game, the focus lies on using the addictive component \parencite{gonzalezGamificationIntelligentTutoring2014}. Often used elements are points, badges, leaderboards and avatars, other mechanisms include content unlocking, storytelling and memes \parencite{zainuddinImpactGamificationLearning2020}.
Often those elements are used specific constellations like the PBL triad described by \textcite{werbachWinHowGame2012}, which contains points, badges and leaderboards.
A system that is not only known from games, but also everyday enterprise features like loyalty programs and employee competitions \parencite{werbachWinHowGame2012}.
Points because they add an absolute scale, badges because they represent a status symbol and work like a temporary goal to strive toward and leaderboards to compare yourself to peers \parencite{werbachWinHowGame2012}.
One of the positive effects of gamification is brought by the feedback in different forms (task, process, self-regulation, self) either immediate or delayed.
Feedback is one of the most important factors in the relation between education and learning \textcite{sailerGamificationLearningMetaanalysis2020}.
The use of gamified elements showed positive outcomes in multiple studies, in general \parencite{hamariDoesGamificationWork2014} as well as in education specific contexts \parencite{sailerGamificationLearningMetaanalysis2020}.
But gamification, especially some elements like leaderboards, can also lead to negative outcomes. Leaderboards, while motivating through comparison, have been reported to demotivate participants \parencite{almeidaSystematicMappingNegative2021}.
"Pavlovication" as \textcite{klabbersArchitectureGameScience2018} calls it, Gamification, as it is often a short question-answer-reward-cycle, conditions the user to learn conditional and narrows the possible ways to solve a problem down ~\parencite{klabbersArchitectureGameScience2018}.
Some studies also suggested that gamified learning platforms also lack individualism regarding choice and display of gamification elements, resulting in discomfort and negative emotions \parencite{santosDoesGenderStereotype2023}.
To combat this missing individualism, \textcite{oliveiraTailoredGamificationEducation2023,dehghanzadehUsingGamificationSupport2024} suggest using more independent variables to taylor the use of gamification elements.