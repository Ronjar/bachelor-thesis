Digital learning environments (DLE) offer a wide range of features and flexibility to support students learning processes. Tutoring systems are one kind of digital learning environments ranging from simple instructive texts to simulations and virtual realities, serving as models that simplify aspects of the real world to reduce complexity mostly from interconnection and context of knowledge for both the machine and the user \parencite{psotkaIntelligentTutoringSystems1988}.
DLE are in use at higher education institutions and widely researched, especially in the field of computer science \parencite{zawacki-richterSystematicReviewResearch2019}.


This could be due to the often mandated transition to remote education during the COVID-19 pandemic highlighted the indispensable role of digital learning environments \parencite{garcia-moralesTransformationHigherEducation2021}.
Especially Tutoring Systems, although often costly and impersonalized have become an important tool for education institutions \parencite{elhadbiReviewStudyAdaptive2024}.
The usage of digital learning resources saw a significant increase during this period; in 2019, 54\% of students utilized such resources, with the figure rising to 70\% in 2020.
In Germany, the percentage of scholars aged ten to fifteen engaging with digital learning materials doubled from 32\% to 64\% within the same timeframe \parencite{statistischesbundesamtDigitalesLernenNimmt2020}.


Tutoring systems can be enhanced with some sort of intelligence, resulting in intelligent tutoring systems (ITS). Intelligent tutoring systems adapt dynamically to learner's needs, incorporating usage factors such as learner performance, but also external factors such as age, culture or gender \parencite{nkambouAdvancesIntelligentTutoring2010, gonzalezGamificationIntelligentTutoring2014}.
The role of artificial intelligence in this process could also become more important in the future, as it could have a significant, yet unknown impact on ITS \parencite{zawacki-richterSystematicReviewResearch2019}.

\textcite{jacksonMotivationPerformanceGamebased2013} found that traditional tutoring systems lead to boredom and disengagement after long periods of use. Further, they showed the use of gamified elements in tutoring systems significantly improved the motivation and performance of students.

%\subsection{Gamification}
Gamification can be defined as "the idea of using game design elements in non-game contexts" \parencite{deterdingGameDesignElements2011} to further increase motivation and user activity within interaction design \parencite{deterdingGameDesignElements2011}.
The concept of gamification also has attracted substantial interest within the educational sciences \parencite{swachaStateResearchGamification2021}.
These game-design elements, subsequently called gamified elements, are elements often found in video games.
However, the concept of gamification is different from designing a game, the focus lies on applying the addictive component to other environments, in this case education \parencite{gonzalezGamificationIntelligentTutoring2014}.
Often used elements are points, badges, leaderboards, avatars, and narrated content. Other mechanisms include content unlocking, storytelling, and memes \parencite{zainuddinImpactGamificationLearning2020}.
Those elements are often used in specific constellations like the PBL triad described by \textcite{werbachWinHowGame2012}, which contains points, badges, and leaderboards.
A system that is not only known from games, but also everyday enterprise features like loyalty programs and employee competitions \parencite{werbachWinHowGame2012}.

\begin{APAitemize}
    \item Points, because they add an absolute scale, allowing for quantifiable measurement of user achievements \parencite{hamariDoesGamificationWork2014}.
    \item Badges, because they represent a status symbol and work like a temporary goal to strive toward, often reflecting mastery or achievement \parencite{gonzalezGamificationIntelligentTutoring2014}.
    \item Leaderboards, to compare oneself to peers, which can motivate through social comparison but may also demotivate if not designed carefully \parencite{hamariDoesGamificationWork2014, almeidaSystematicMappingNegative2021}.
    \item Avatars, as they allow users to customize their virtual representation, enhancing their identification with the activity and increasing engagement \parencite{gonzalezGamificationIntelligentTutoring2014}.
    \item Narrated content, which uses storytelling to provide context to activities, thus enriching the user's experience by embedding tasks within an appealing story \parencite{gonzalezGamificationIntelligentTutoring2014}.
\end{APAitemize}
Narrated content or storytelling and avatars are particularly interesting as there appears to be comparatively little research available on these elements, unlike the more extensively studied points, badges, and leaderboards, which was observed during the literature review.
One of the positive effects of gamification is brought by the feedback in different forms (task, process, self-regulation, self) either immediate or delayed.
Feedback is one of the most important factors in the relation between education and learning \parencite{sailerGamificationLearningMetaanalysis2020}.
The use of gamified elements as they provide feedback constantly \parencite{woutersMetaanalysisCognitiveMotivational2013} showed positive outcomes in multiple studies, in general \parencite{hamariDoesGamificationWork2014} as well as in education specific contexts \parencite{sailerGamificationLearningMetaanalysis2020}.


In the realm of computer science, the application of gamified elements is well-documented, demonstrating a important presence due to the inherent integration of technology in the field \parencite{dichevGamifyingEducationWhat2017}.
Further, gamified elements are often included into DLE, as they significantly improve the learning experience \parencite{dermevalGaTOOntologicalModel2019}, motivation and enjoyment \parencite{gonzalezGamificationIntelligentTutoring2014, jacksonMotivationPerformanceGamebased2013}.
Incorporating gamified elements not only enhances the engagement and motivation within the DLE but also necessitates mechanisms for tracking progress, such as content unlocking \parencite{gonzalezGamificationIntelligentTutoring2014}.
The evolving landscape of DLE research also includes emotional and relational dynamics, linking student emotions and teacher-student relationships to learning efficacy and motivation \parencite{woolfAffectiveTutorsAutomatic2010}.
%These insights have led to the development of digital companions, often named pedagogical agents, within DLE that significantly boost the learning potential and self-concept of students, particularly those who are low-achieving.


However, despite the potential positive impacts, the introduction of gamified elements can also lead to negative outcomes.
\textcite{almeidaSystematicMappingNegative2021} found in their systematic mapping, 77 papers mentioned negative effects like cheating, lack of understanding, demotivation in leaderboards and most often lack of effect of the gamified elements.
Moreover, "Pavlovication" as \textcite{klabbersArchitectureGameScience2018} calls Gamification, is often a short question-answer-reward-cycle, conditions the user to learn conditional and narrows the possible ways to solve a problem down \parencite{klabbersArchitectureGameScience2018}.
Some studies also suggested that gamified learning platforms also lack individualism regarding choice and display of gamification elements, resulting in discomfort and negative emotions \parencite{santosDoesGenderStereotype2023}.
To combat this missing individualism, \textcite{oliveiraTailoredGamificationEducation2023,dehghanzadehUsingGamificationSupport2024} suggest using more independent variables to tailor the use of gamification elements.
Multiple meta analyses came to the conclusion, further investigation on the effects of different gamification elements on individuals is needed \textcite{oliveiraTailoredGamificationEducation2023,dehghanzadehUsingGamificationSupport2024,hamariDoesGamificationWork2014}, as future systems could also use more individualized data to further enhance the experience of the gamified elements inside the tutoring system.
Factors mentioned by \textcite{dehghanzadehUsingGamificationSupport2024} include gender and age, \textcite{oliveiraTailoredGamificationEducation2023} mentions culture and gender. All studies also highlight the importance of the context, in which the learner is exposed to gamified elements.
For example, technological proficiency and communication challenges in online settings which have been identified as factors affecting learning satisfaction, with older students showing a stronger preference for face-to-face learning, which may be due to less familiarity with digital technologies \parencite{dabajRoleGenderAge2009}.


Gender, as a concept within social sciences, refers to more than the binary categorization of male and female.
It encompasses a range of identities and experiences that are shaped by a complex interplay of biological, psychological, and social factors.
Gender is not solely determined by biological characteristics; instead, it is increasingly recognized as a spectrum, acknowledging the presence of diverse gender identities beyond the traditional binary understanding \parencite{lindqvistWhatGenderAnyway2021}.
Socialization plays a critical role in shaping gender identity. It influences how individuals perceive themselves and interact with their surroundings based on the gender norms prevalent within their society.
These norms dictate behaviours, roles, and expectations, which are often internalized from an early age through various socialization agents like family, media, educational institutions, and peer groups \parencite{kampshoffHandbuchGeschlechterforschungUnd2012}.
While acknowledging the spectrum of gender identities, this thesis will focus primarily on the binary categorization of gender—male and female.
This approach does not negate the validity of non-binary or genderqueer identities but rather limits the scope of investigation to traditional gender roles within the binary framework.

Gender is a critical variable to consider in DLE as research indicates that male and female students exhibit distinct preferences for learning modalities and react differently to adaptive learning technologies.
For instance, studies have shown that while male students often prefer multimodal instructional approaches, female students tend to favour single-mode learning, particularly kinesthetic styles \parencite{wehrweinGenderDifferencesLearning2007}.
Additionally, the use of adaptive learning technologies has demonstrated a more pronounced improvement in performance among male students compared to their female counterparts in subjects like Mathematics and Portuguese \parencite{desantanaEvaluatingImpactMars2016}.
Recognizing and addressing gender differences in learning preferences is essential for tailoring educational technologies and strategies, thereby optimizing tutorial systems to enhance learning efficacy and engagement for all students.

The design of virtual classroom environments significantly influences gender disparities in computer science courses, impacting both course selection and anticipated success.
Research by \textcite{cheryanClassroomsMatterDesign2011} shows that altering the design of virtual classrooms from stereotypical computer science environments  to more neutral or non-stereotypical settings (e.g., featuring art, nature posters) can substantially increase women's interest and perceived success in computer science.
This change in environment reduces the gender gap by creating a greater sense of belonging among female students, which is not as pronounced in male students.


A study noted that ITS programs with a male companion were muted twice as often as those with a female companion, highlighting potential gender differences that could be explored to enhance the predictive capabilities of the ITS \parencite{woolfAffectiveTutorsAutomatic2010}.
Competition, often created with leaderboards, results in higher gratification in men compared to women \parencite{lucasSexDifferencesVideo2004}.
Gamified environments also have been reported to increase tension and lower perceived competence in women compared to an perceived competence increase in men \parencite{laiszpedroDoesGamificationWork2015}.

This can also be seen in a study by \textcite{albuquerqueDoesGenderStereotype2017} exploring the impact of gender stereotype threats in gamified educational environments.
Their research demonstrated that male-dominated gamified contexts significantly increased anxiety levels among participants.
This was particularly evident among female participants, who reported increased anxiety in such environments, potentially affecting their academic performance and engagement.
The study employed a methodologically robust design involving a pretest-posttest setup and a gamified logic quiz to simulate a learning environment.
\textcite{albuquerqueDoesGenderStereotype2017} underline the critical influence of gender considerations in the design of educational technologies, emphasizing the need to create inclusive environments that do not inadvertently perpetuate gender biases. 

\subsection{This Study}
This thesis aims to explore the effects of different gamified elements and their combinations on performance and anxiety in a digital learning environment, especially focusing on the influence of gender on these effects.
Gender could be of particular interest, as it is a factor that is often known beforehand and already has shown to be a great influence in learning.

As such, our study is based on the work of \textcite{albuquerqueDoesGenderStereotype2017}.
Although similar, the study did focus on stereotype threat and did not investigate the effects of different gamified elements on performance and anxiety.

Also important in this and \textcite{albuquerqueDoesGenderStereotype2017} study's context, males perform better than females in solving progressive matrices from age 15 onward \parencite{lynnSexDifferencesProgressive2004}.

This thesis will assess whether different gamified elements can enhance the learning experience without disadvantaging any gender group.
The primary goal is to design digital tutoring systems that fully leverage the potential of gamification to benefit all users equally, fostering an inclusive and effective educational environment.
