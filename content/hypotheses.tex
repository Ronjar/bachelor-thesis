\section{Hypotheses}

Wie in dem ersten Kapitel zu erkennen, gibt es offene Fragen zu der Effizienz verschiedener gamifizierter Elemente und dem Bezug verschiedener Geschlechter zu diesen gamifizierten Elementen.
Generell ist die Frage nach der Effizienz bestimmter Elemente und Elementkombinationen noch zu klären \parencite{dehghanzadehUsingGamificationSupport2024}.
Um die Verbindung von Geschlecht und Gamification-Elementen zu untersuchen, haben wir folgendes Modell erstellt:
**Modell einfügen**

\begin{tikzpicture}[node distance=0cm]
    \tikzstyle{startstop} = [rectangle, rounded corners, minimum width=3cm, minimum height=1cm,text centered, draw=black, fill=red!30]
    \tikzstyle{process} = [rectangle, minimum width=3cm, minimum height=1cm, text centered, draw=black, fill=orange!30]
    \tikzstyle{arrow} = [thick,->,>=stealth]

    \node (gender) [process] {Gender};
    \node (gamified) [startstop, below left=2cm and 2cm of gender] {Gamified Elements};
    \node (anxiety) [process, below right=1.4cm and 2cm of gender] {Anxiety};
    \node (performance) [process, below=0.2cm of anxiety] {Performance};

    \coordinate (MidPoint1) at ($(anxiety.west)!0.5!(performance.west)$);

    % Calculate midpoint for the horizontal arrow
    \coordinate (MidPoint2) at ($(gamified.east)!0.5!(MidPoint1)$);

    \draw [arrow] (gamified) -- (MidPoint1);
    \draw [arrow] (gender) -- (MidPoint2);
\end{tikzpicture}